%
% Portuguese-BR vertion
% 
\documentclass{article}

\usepackage{ipprocess}
% Use longtable if you want big tables to split over multiple pages.
% \usepackage{longtable}
\usepackage[utf8]{inputenc} 
% Babel package is used to translate keywors to a specific language. 
% The option "brazil" defines portuguese-brazil as default language.
% Babel is also useful for hiphenation.
\usepackage[brazil]{babel}
\usepackage[T1]{fontenc}
\usepackage{tikz}
\pagestyle{fancy}
\usepackage{libertine}

\sloppy

\title{Unidade de Operações Aritméticas}

\graphicspath{{./pictures/}} % Pictures dir
\makeindex
\begin{document}

\capa{1.0}{}{2014}{}{Especificação de Requisitos}{Universidade Estadual de Feira de Santana}
\newpage

%%%%%%%%%%%%%%%%%%%%%%%%%%%%%%%%%%%%%%%%%%%%%%%%%%
%% Revision History
%%%%%%%%%%%%%%%%%%%%%%%%%%%%%%%%%%%%%%%%%%%%%%%%%%
\section*{\center Histórico de Revisões}
  \vspace*{1cm}
  \begin{table}[ht]
    \centering
    \begin{tabular}[pos]{|m{2cm} | m{7.2cm} | m{3.8cm}|} 
      \hline
      \cellcolor[gray]{0.9}
      \textbf{Date} & \cellcolor[gray]{0.9}\textbf{Descrição} & \cellcolor[gray]{0.9}\textbf{Autor(s)}\\ \hline
      \hline
      \small 11/08/2014 & \small Concepção do documento & \small João Bittencourt \\ \hline      
      \small 12/08/2014 &
      \begin{small}
        \begin{itemize}
          \item Inclusão do requisito [FR2.12];
          \item Substituição do requisito [FR2.22];
          \item Remoção do requisito [FR2.43];
          \item Inclusão do requisito [NFR1];
        \end{itemize}
      \end{small} & \small João Bittencourt e\newline Anfranserai Dias \\ \hline 
    \end{tabular}
  \end{table}

\newpage

% TOC instantiation
\tableofcontents
\newpage

%%%%%%%%%%%%%%%%%%%%%%%%%%%%%%%%%%%%%%%%%%%%%%%%%%
%% Document main content
%%%%%%%%%%%%%%%%%%%%%%%%%%%%%%%%%%%%%%%%%%%%%%%%%%
\section{Introdução}

\subsection{Visão Geral do Documento}
  \begin{itemize}
   \item \textbf{Requisitos funcionais -} lista de todos os requisitos funcionais.
   \item \textbf{Requisitos não funcionais -} lista de todos os requisitos não funcionais.
   \item \textbf{Dependências -} conjunto de dependências de IP-cores previstos.
   % \item \textbf{Notas -} apresenta a lista de notas apresentadas ao longo do documento.
   % \item \textbf{Referências -} lista de todos os textos referenciados nesse documento.
  \end{itemize}

  \subsection{Acrônimos e Abreviações}
  
  \FloatBarrier
  \begin{table}[H]
    \begin{center}
      \begin{tabular}[pos]{|m{5cm} | m{9cm}|} 
        \hline
        \cellcolor[gray]{0.9}\textbf{Termo} & \cellcolor[gray]{0.9}\textbf{Descrição} \\ \hline
        Requisitos Funcionais     & Requisitos de hardware que compõem os módulos, descrevendo as ações que o 
                                    mesmo deve estar apto a executar. Estas informações são capturadas a partir 
                                    do desenvolvimento dos casos de uso, que documentam as entradas, os processos 
                                    e as saídas geradas.  \\ \hline
        Requisitos Não Funcionais & Requisitos de hardware que compõem os módulos, representando as características 
                                    que o mesmo deve ter, ou restrições que o mesmo deve operar. Estas características
                                    referem-se a técnicas, algoritmos, tecnologias e especificidades do Sistema como um todo.  \\ \hline
        Dependências              & Requisitos de reuso de IP-cores, descrevendo as funções que cada um deve exercer. \\ \hline
      \end{tabular}
    \end{center}
  \end{table}  

  \subsection{Prioridades dos Requisitos}
    Os requisitos apresentados nesse documento são classificados em três categorias.

    \begin{description}
      \item[Importante:] requisito sem o qual o sistema funciona, porém não como deveria.
      \item[Essencial:] requisito que deve ser implementado para que o sistema funcione.
      \item[Desejável:] requisito que não compromete o funcionamento do sistema.
    \end{description}

\section{Requisitos Funcionais}

  \subsection{Interface de Comunicação RS232}
  
    \begin{functional}
      \item \textbf{Taxa de velocidade de transmissão e recepção (\textit{baud rate})} \hrule
      O processo de comunicação seria deve ocorrer à uma taxa de transmissão igual a $9600$ bps.

      \textbf{Prioridade: Essencial}

      \item \textbf{Modo de operação do controlador RS232} \hrule
      O controlador serial RS232 opera no modo 8-N-1.

      \textbf{Prioridade: Essencial}      

      \item \textbf{Frequência de operação do circuito} \hrule
      A frequência de clock do sistema deve ser definida de acordo com a taxa de frequência do oscilador principal do kit FPGA utilizado para prototipação.
     
      \textbf{Prioridade: Essencial}   

      \item \textbf{Padrão de sequência de pacotes para uma operação} \hrule
      Uma operação é representada pela transmissão de três pacotes de dados na ordem que se segue: (1) código da operação, (2) primeiro operando e (3) segundo operando.

      \textbf{Prioridade: Essencial}       

    \end{functional}

  \subsection{Interface de Saída}
  
    \begin{functional}
     \item \textbf{Mecanismo de armazenamento de resultado}\hrule
     O resultado de uma operação deve ser armazenado em um registrador interno.

     \textbf{Prioridade: Importante}

      \item \textbf{Protocolo de saída de dados do resultado}\hrule
      Os dados devem ser apresentado em sua codificação binária, representada através de um conjunto de LEDs.
     
     \textbf{Prioridade: Essencial} 

    \end{functional}  

  \subsection{Conjunto de Operações Aritméticas} 
  
    \begin{functional}
      \item \textbf{Operação de soma}\hrule
      O módulo deve ser capaz de realizar a operação de soma de dois valores de 8 bits.

      \textbf{Prioridade: Essencial}

      \item \textbf{Operação de subtração}\hrule
      O módulo deve ser capaz de realizar a operação de subtração de dois valores de 8 bits.

      \textbf{Prioridade: Essencial}

      \item \textbf{Operação de multiplicação}\hrule
      O módulo deve ser capaz de realizar a operação de multiplicação de dois valores de 8 bits. 

      \textbf{Prioridade: Essencial}

      \item \textbf{Operação de divisão}\hrule
      O módulo deve ser capaz de realizar a operação de divisão de dois valores de 8 bits.

      \textbf{Prioridade: Essencial}      

      \item \textbf{Tamanho da palavra de saída}\hrule
      O componente deve apresentar uma saída única de 8 bits para todas as operações aritméticas.

     \item \textbf{Detecção de overflow aritmético}\hrule
     O módulo deve ser capaz de detectar \textit{overflow} aritmético.

     \textbf{Prioridade: Importante}
    \end{functional}

  \subsection{Conjunto de Operações Lógicas} 
  
    \begin{functional}
      \item \textbf{Operação AND}\hrule
      O módulo deve ser capaz de realizar a operação AND lógico de dois operandos de 8 bits.

      \textbf{Prioridade: Essencial}

      \item \textbf{Operação OR}\hrule
      O módulo deve ser capaz de realizar a operação OR lógico de dois operandos de 8 bits.

      \textbf{Prioridade: Essencial}      
     
    \end{functional}    

  \section{Requisitos não Funcionais}
  Esta seção apresenta a lista de Requisitos não Funcionais do projeto.

  \begin{nonfunctional}
    \setcounter{enumi}{0} % turn it transparent to the writer
     \item \textbf{Transmissão dos dados via porta serial}\hrule
     Os dados devem ser transmitidos através de uma interface de software dedicada.
  
     \textbf{Prioridade: Importante}

  \end{nonfunctional}

\section{Dependências}
Esta seção apresenta uma lista dos soft IP-cores disponíveis para reuso e que devem ser adotados no desenvolvimento deste projeto.

\textbf{Prioridade: Essencial}

  \begin{dependencies}
    \item \textbf{Controlador RS232}\hrule
    Módulo de comunicação serial via protocolo RS232 disponível no repositório OpenCores.

\end{dependencies}  

% Optional bibliography section
% To use bibliograpy, first provide the ipprocess.bib file on the root folder.
% \bibliographystyle{ieeetr}
% \bibliography{ipprocess}

\end{document}
